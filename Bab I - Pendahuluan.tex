% ==========================================
% BAB I PENDAHULUAN
% ==========================================
\chapter{PENDAHULUAN}
\label{chap:pendahuluan}

Bab ini membahas secara umum permasalahan yang menjadi dasar penyusunan tugas akhir. Bab ini bertujuan untuk memberikan gambaran awal dan pemahaman mendasar mengenai topik yang diangkat. Pada subbab latar belakang dijelaskan faktor-faktor yang menimbulkan permasalahan. Selanjutnya, subbab rumusan masalah, tujuan, serta batasan masalah digunakan untuk memperjelas dan membatasi ruang lingkup permasalahan dengan konkrit. Terakhir, subbab metodologi membahas tentang proses perumusan solusi.

% --- Latar Belakang ---
\section{Latar Belakang}

Suku bunga acuan Bank Indonesia, yang saat ini dikenal dengan nama 
BI-Rate, merupakan instrumen kebijakan moneter yang digunakan sebagai 
sinyal kebijakan untuk mengendalikan ekspektasi inflasi dan menjaga 
stabilitas nilai rupiah. Dalam perkembangannya, Bank Indonesia 
melakukan penguatan kerangka operasi moneter pada 19 Agustus 2016 
dengan mengimplementasikan BI-7 Day Reverse Repo Rate (BI7DRR). 
Instrumen ini dipilih karena bersifat transaksional di pasar uang, 
sehingga memiliki hubungan yang lebih kuat dalam memengaruhi perbankan 
dan sektor riil. Selanjutnya, mulai 21 Desember 2023, Bank Indonesia 
kembali menggunakan nama BI-Rate menggantikan BI7DRR. Perubahan nama 
ini bertujuan untuk memperkuat komunikasi kebijakan moneter, tanpa 
mengubah makna, tujuan, maupun operasionalisasinya yang tetap mengacu 
pada transaksi \textit{reverse repo} Bank Indonesia tenor 7 hari \autocite{bi2023moneter}.

Penetapan BI-Rate dilakukan melalui Rapat Dewan Gubernur (RDG) Bank 
Indonesia dengan mempertimbangkan berbagai indikator makroekonomi 
yang kompleks, baik indikator domestik maupun global. Secara teoritis, 
keputusan ini sering berlandaskan pada konsep \textit{Taylor Rule}, 
yang menyarankan bank sentral harus menyesuaikan suku bunga nominal 
sebagai respon deviasi inflasi aktual dari targetnya dan kesenjangan 
\textit{output} ekonomi \autocite{taylor1993discretion}. Namun, 
seiring berkembangnya dinamika ekonomi global, variabel yang 
memengaruhi keputusan kebijakan moneter semakin bertambah dan rumit, 
jauh melampaui Taylor Rule awal yang cukup sederhana. Mekanisme 
transmisi kebijakan moneter sangatlah kompleks, sehingga melandaskan 
keputusan hanya pada fungsi sederhana dari inflasi dan \textit{output gap} 
menjadi tidak relevan dalam praktik bank sentral modern \autocite{svensson2003what}.

Secara historis, pemodelan ekonomi di lingkungan bank sentral sangat 
bergantung pada pendekatan ekonometrika tradisional untuk 
memproyeksikan variabel makroekonomi. Model statistik konvensional 
seperti \textit{autoregressive} (AR) dan \textit{vector autoregressive} (VAR) 
telah lama menjadi standar dalam analisis deret waktu untuk menangkap 
hubungan antar variabel \autocite{chakraborty2017machine}. Namun, 
metode konvensional ini memiliki keterbatasan fundamental karena 
umumnya berasumsi pada hubungan linear, sehingga sering kali kesulitan 
menangkap pola non-linier yang kompleks yang melekat pada data 
keuangan modern \autocite{sezer2020financial}. Ketidakmampuan model 
tradisional dalam menangkap dependensi non-linier ini dapat menyebabkan 
stagnasi atau penurunan akurasi prediksi, terutama ketika dihadapkan 
pada dataset dengan volatilitas tinggi.

Keterbatasan model tradisional ini mendorong pergeseran ke arah metode 
yang lebih modern. Pendekatan \textit{machine learning} (ML) dan \textit{deep learning} (DL) 
terbukti lebih andal dalam menangkap pola non-linier \autocite{chen2016xgboost}. 
Algoritma ML mampu menangkap struktur data keuangan yang kompleks 
dengan lebih baik. Model \textit{deep learning} seperti \textit{Long Short-Term Memory} (LSTM) 
mampu mengurangi tingkat kesalahan prediksi secara signifikan 
dibandingkan model ARIMA \autocite{sezer2020financial}. Kemampuan 
LSTM untuk mempelajari ketergantungan temporal jangka panjang 
menjadikannya relevan untuk menangani data ekonomi yang fluktuatif \autocite{hochreiter1997long}.

Walau model ML dan DL menunjukkan akurasi prediksi yang tinggi, 
tantangan utama yang dihadapi adalah interpretabilitas. Model ML 
dan DL beroperasi sebagai \textit{black box} yang menyulitkan 
pemahaman terhadap faktor-faktor yang mendasari hasil prediksi \autocite{rudin2019stop}. 
Dalam konteks kebijakan ekonomi, akurasi semata tidak cukup, pembuat 
kebijakan membutuhkan pemahaman kausal mengenai faktor yang 
memengaruhi prediksi demi akuntabilitas publik \autocite{bussmann2021explainable}. 
Bank of England menyatakan bahwa penggunaan ML di bank sentral wajib 
disertai transparansi agar keputusan yang diambil memiliki legitimasi \autocite{chakraborty2017machine}. 
Sesuai pernyataan tersebut, sektor keuangan memerlukan model yang 
dapat dijelaskan (\textit{explainable}) untuk memenuhi standar 
regulasi dan manajemen risiko \autocite{bussmann2021explainable}.

Untuk menjawab kebutuhan tersebut, pendekatan \textit{explainable artificial intelligence} (XAI) 
seperti \textit{SHapley Additive exPlanations} (SHAP) berkembang pesat 
sebagai solusi untuk melihat kontribusi variabel dalam model kompleks \autocite{adadi2018peeking}. 
Penerapan ML di bank sentral global telah meluas, mulai dari analisis 
makroekonomi hingga stabilitas finansial \autocite{chakraborty2017machine}. 
Studi terbaru menunjukkan model berbasis \textit{Gradient Boosting} 
yang diinterpretasikan dengan SHAP dapat menangkap faktor risiko 
finansial secara akurat dan tetap transparan \autocite{moscatelli2020corporate}.

Namun, di Indonesia, penelitian prediksi BI-Rate yang mengintegrasikan 
ML/DL dengan XAI masih sangat terbatas. Di sisi lain, meskipun solusi 
XAI tersedia, hasil outputnya masih berupa grafik teknis atau matriks 
angka. Oleh karena itu, penelitian ini hadir untuk mengisi gap 
tersebut dengan mengintegrasikan \textit{Large Language Models} (LLM) 
sebagai bantuan kognitif. Sebagaimana disarankan dalam riset prediksi 
finansial terkini, LLM tidak hanya berfungsi sebagai penerjemah teks, 
tetapi mampu menarasikan logika kausalitas dari interpretasi teknis XAI 
menjadi wawasan kebijakan yang intuitif dan komunikatif, sehingga 
hasil prediksi dapat langsung dipahami oleh pemangku kepentingan 
non-teknis dalam perumusan kebijakan moneter \autocite{yu2023temporal}.

% --- Rumusan Masalah ---
\section{Rumusan Masalah}

Berdasarkan latar belakang yang telah dibahas, terdapat kebutuhan akan 
model prediksi BI-Rate yang akurat sekaligus transparan, untuk 
mendukung terciptanya kebijakan ekonomi yang tepat. Hal tersebut 
melahirkan beberapa permasalahan sebagai berikut:

\begin{enumerate}
\item Bagaimana membangun model prediksi BI-Rate yang mampu menangkap hubungan linear maupun non-linier antar indikator makroekonomi menggunakan pendekatan ML dan DL?
\item Bagaimana perbandingan performa akurasi antara model ekonometrika tradisional, ML, dan DL dalam memprediksi BI-Rate?
\item Bagaimana penerapan teknik XAI untuk memperlihatkan kontribusi indikator makroekonomi terhadap hasil prediksi model secara transparan?
\item Bagaimana memanfaatkan LLM untuk menerjemahkan hasil interpretasi teknis dari XAI menjadi penjelasan naratif yang dapat dipahami dalam konteks kebijakan ekonomi?
\end{enumerate}

% --- Tujuan ---
\section{Tujuan}

Berdasarkan rumusan masalah yang sudah dijelaskan, didapatkan tujuan 
dari tugas akhir sebagai berikut. Tujuan yang dibuat menjawab 
pertanyaan yang muncul dalam rumusan masalah.

\begin{enumerate}
\item Mengembangkan model prediksi BI-Rate berbasis ML dan DL dengan memanfaatkan data indikator makroekonomi domestik dan global.
\item Melakukan evaluasi komparatif terhadap kinerja model yang dikembangkan dibandingkan dengan model tradisional berdasarkan metrik \textit{error} prediksi.
\item Mengimplementasikan teknik XAI untuk memberikan interpretasi global dan lokal dari hasil prediksi yang dihasilkan oleh model prediktif.
\item Menghasilkan mekanisme pelaporan otomatis berbasis LLM yang mampu menarasikan alasan di balik prediksi BI-Rate berdasarkan analisis fitur XAI.
\end{enumerate}

% --- Batasan Masalah ---
\section{Batasan Masalah}

Dalam pengerjaan tugas akhir ini dibutuhkan batasan masalah agar lingkup dari tugas akhir yang dikerjakan tidak terlalu luas. Berdasarkan tujuan yang sudah dijelaskan, dapat dibuat batasan masalah sebagai berikut.

\begin{enumerate}
\item Dataset yang digunakan bersumber dari data resmi Bank Indonesia, Badan Pusat Statistik (BPS), Bank for International Settlements (BIS) Data Portal, Federal Reserve Economic Data (FRED), dan sumber kredibel lainnya. Data mencakup periode waktu bulanan dari Juli 2005 hingga September 2024.
\item Penggunaan LLM dalam penelitian ini difokuskan hanya sebagai pembuat narasi untuk menjelaskan hasil interpretasi XAI. LLM tidak digunakan untuk melakukan kalkulasi prediksi angka suku bunga secara langsung.
\end{enumerate}

% --- Metodologi Pengerjaan TA ---
\section{Metodologi}

Dalam pengerjaan tugas akhir ini, dibutuhkan alur pengerjaan yang 
terstruktur supaya tujuan yang telah dirumuskan dapat tercapai. Metodologi 
tugas akhir yang digunakan adalah CRISP-DM (\textit{Cross-Industry Standard Process for Data Mining}) 
yang merupakan kerangka kerja standar yang bersifat independen 
terhadap industri dan telah menjadi acuan utama dalam proyek \textit{data mining} \autocite{schroer2021systematic}. 
Pemilihan CRISP-DM didasarkan pada kemampuannya dalam mengelola siklus 
proyek berbasis data dan telah terbukti banyak digunakan di berbagai 
sektor \autocite{martinezplumed2021crisp}. Penulis tidak melakukan 
tahap \textit{deployment} pada tugas akhir ini karena fokusnya adalah 
pada pengembangan dan evaluasi model prediksi, bukan penerapan sistem 
secara operasional.

\begin{figure}[h]
  \centering
  \captionsetup{justification=centering}
      \includegraphics[width=0.6\textwidth]{image/crispdm.png}
  \caption{Metodologi CRISP-DM \autocite{martinezplumed2021crisp}}
  \label{gambar:crispd}
\end{figure}

Tahapan dalam CRISP-DM dijelaskan sebagai berikut:

\begin{enumerate}

\item \textit{Business Understanding}

Pada tahap ini dilakukan identifikasi kebutuhan dan tujuan tugas akhir dalam konteks kebijakan moneter Indonesia. Tahap ini juga meliputi eksplorasi literatur untuk membantu perumusan masalah, objektif, dan parameter keberhasilan tugas akhir ini.

\item \textit{Data Understanding}

Pada tahap \textit{data understanding} dilakukan pengumpulan dan eksplorasi awal data makroekonomi dari sumber terpercaya, serta melakukan analisis struktur data, kelengkapan, pola dasar, identifikasi potensi masalah data, dan mencari \textit{insights} yang mungkin belum didapatkan sebelumnya.

\item \textit{Data Preparation}

Pada tahap \textit{data preparation}, data dipersiapkan agar layak digunakan untuk proses pemodelan. Tahap ini meliputi pemrosesan data, transformasi format, dan konstruksi fitur yang diperlukan untuk meningkatkan kualitas \textit{input} model. \textit{Dataset} akhir merupakan data yang sudah layak dan siap untuk \textit{training} dan pengujian model.

\item \textit{Modelling}

Pada tahap \textit{modelling}, dilakukan pengembangan model prediksi, konfigurasi parameter, pelatihan model, dan validasi model prediksi yang sesuai terhadap data terkait.

\item \textit{Evaluation}

Pada tahap \textit{evaluation}, hasil model dan aspek interpretabilitas dievaluasi berdasarkan metrik yang telah ditentukan untuk menilai apakah model yang dikembangkan telah memenuhi tujuan awal tugas akhir dan layak dinyatakan selesai. Jika hasil belum memuaskan, dapat dilakukan iterasi kembali ke tahap sebelumnya untuk perbaikan, baik data maupun model.

\end{enumerate}