% ==========================================
% BAB V RENCANA SELANJUTNYA
% ==========================================
\chapter{RENCANA SELANJUTNYA}
\label{chap:rencana-selanjutnya}

Pada bab ini dijelaskan terkait rencana pengerjaan tugas akhir, termasuk perkakas dan infrastruktur yang diperlukan, jadwal kerja, alokasi anggaran, desain pengujian dan evaluasi sistem, serta identifikasi dan mitigasi risiko yang berpotensi terjadi selama proses pengerjaan tugas akhir ini. Perencanaan ini dilakukan untuk memastikan kelancaran eksekusi tugas akhir dalam mencapai tujuan yang telah ditetapkan.

% --- Rencana Implementasi ---
\section{Rencana Implementasi}
\label{sec:rencana-implementasi}

Pada subbab ini dijelaskan rencana teknis pelaksanaan penelitian, mencakup perkakas yang digunakan, serta linimasa pengerjaan tugas akhir.

\subsection{Perkakas yang Dibutuhkan}
Rincian perkakas yang dibutuhkan untuk mengerjakan tugas akhir ini dapat diliaht pada Tabel \ref{tbl:perkakas-dibutuhkan}.

\input table/Tabel_Perkakas_Dibutuhkan.tex

\subsection{Linimasa Pengerjaan Tugas Akhir}
Linimasa pengerjaan tugas akhir secara umum dapat dilihat pada Tabel \ref{tbl:linimasa-pengerjaan}.

\input table/Tabel_Linimasa_Pengerjaan.tex

% --- Rencana Anggaran Biaya ---
\section{Rencana Anggaran Biaya}
\label{sec:rencana-anggaran}

Rincian estimasi biaya anggaran untuk mengerjakan tugas akhir ini dapat dilihat pada Tabel \ref{tbl:rencana-anggaran}.

\input table/Tabel_Rencana_Anggaran.tex

% --- Desain Pengujian dan Evaluasi ---
\section{Desain Pengujian dan Evaluasi}
\label{sec:desain-pengujian}

Pengujian dan evaluasi sistem dilakukan melalui dua aspek utama, yaitu verifikasi kinerja model secara teknis dan validasi dari perspektif ekonomi.

\subsection{Verifikasi Kinerja Model}
Seluruh skenario verifikasi kinerja model dapat dilihat pada Tabel \ref{tbl:skenario-verifikasi}.

\input table/Tabel_Skenario_Verifikasi.tex

\subsection{Validasi Ekonomi}
Seluruh skenario validasi ekonomi ini akan dilakukan kepada ahli di bidang terkait (ekonomi) dan rincian skenarionya dapat dilihat pada Tabel \ref{tbl:skenario-validasi}.

\input table/Tabel_Skenario_Validasi.tex

% --- Analisis Risiko dan Mitigasi ---
\section{Analisis Risiko dan Mitigasi}
\label{sec:analisis-risiko}

Terdapat beberapa potensi risiko dalam tugas akhir ini yang dapat memengaruhi hasil akhir, beserta upaya mitigasi yang telah dirancang untuk mengantisipasinya. Rincian lengkap analisis risiko dan mitigasi dapat dilihat pada Tabel \ref{tbl:analisis-risiko}.

\input table/Tabel_Analisis_Risiko.tex

Penerapan langkah-langkah di atas diharapkan dapat meminimalisir berbagai risiko yang terjadi, sehingga proses Tugas Akhir dapat berlangsung secara optimal sesuai dengan perencanaan.