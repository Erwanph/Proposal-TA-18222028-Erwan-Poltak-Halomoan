% ==========================================
% TABEL PERBANDINGAN KELEBIHAN DAN KEKURANGAN
% ==========================================
\begin{table}[H]
\centering
\caption{Perbandingan Kelebihan dan Kekurangan Alternatif Solusi}
\label{tbl:perbandingan-solusi}
\begin{tabular}{|p{3cm}|p{5cm}|p{5cm}|}
\hline
\textbf{Alternatif Solusi} & \textbf{Kelebihan} & \textbf{Kekurangan} \\
\hline
Model \textit{Machine Learning} dengan XAI dan Narasi LLM & 
\begin{itemize}[leftmargin=*, nosep, topsep=0pt]
\item Akurasi prediksi tinggi pada data non-linear.
\item Transparansi lengkap (visual \& teks).
\item Sangat mudah dipahami (\textit{User-friendly}).
\end{itemize} & 
\begin{itemize}[leftmargin=*, nosep, topsep=0pt]
\item Kompleksitas implementasi teknis tinggi.
\item Ketergantungan pada API pihak ketiga.
\end{itemize} \\
\hline
Model Ekonometrik Struktural & 
\begin{itemize}[leftmargin=*, nosep, topsep=0pt]
\item Landasan teori ekonomi kuat.
\item Komputasi ringan dan cepat.
\end{itemize} & 
\begin{itemize}[leftmargin=*, nosep, topsep=0pt]
\item Akurasi prediksi terbatas.
\item Tidak ada narasi otomatis.
\item Interpretasi sulit bagi awam.
\end{itemize} \\
\hline
Model \textit{Machine Learning} dengan Visualisasi XAI & 
\begin{itemize}[leftmargin=*, nosep, topsep=0pt]
\item Akurasi prediksi tinggi.
\item Transparan secara visual (grafik).
\end{itemize} & 
\begin{itemize}[leftmargin=*, nosep, topsep=0pt]
\item Grafik XAI masih terlalu teknis.
\item Membutuhkan interpretasi manual.
\item \textit{User experience} kurang ramah bagi awam.
\end{itemize} \\
\hline
\end{tabular}
\end{table}
