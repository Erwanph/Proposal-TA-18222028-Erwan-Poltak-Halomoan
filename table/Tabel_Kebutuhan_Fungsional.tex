% ==========================================
% TABEL KEBUTUHAN FUNGSIONAL
% ==========================================
\begin{table}[H]
\centering
\caption{Kebutuhan Fungsional}
\label{tbl:kebutuhan-fungsional}
\begin{tabular}{|p{1cm}|p{5cm}|p{7cm}|}
\hline
\textbf{Kode} & \textbf{Kebutuhan Fungsional} & \textbf{Deskripsi} \\
\hline
F-01 & \textit{Input} Data & Sistem mampu menerima \textit{input} data indikator makroekonomi yang akan dilatih melalui \textit{dataset}. \\
\hline
F-02 & Prediksi BI-Rate & Sistem mampu memproses data input dan menghasilkan prediksi nilai BI-Rate. \\
\hline
F-02 & Interpretasi Fitur & Sistem mampu menghitung dan menampilkan skor kontribusi dari setiap variabel input terhadap hasil prediksi. \\
\hline
F-03 & Visualisasi Data Interaktif & Sistem mampu menampilkan grafik tren historis dan plot interpretasi XAI untuk memudahkan analisis visual pengguna. \\
\hline
F-04 & Generasi Narasi Otomatis & Sistem mampu mengonversi hasil prediksi dan interpretasi menjadi teks narasi kebijakan menggunakan LLM. \\
\hline
\end{tabular}
\end{table}
