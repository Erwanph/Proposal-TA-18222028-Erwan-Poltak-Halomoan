% ==========================================
% TABEL KEBUTUHAN NON-FUNGSIONAL
% ==========================================
\begin{longtable}{|p{1.5cm}|p{4cm}|p{7.5cm}|}
\caption{Kebutuhan Non-Fungsional}
\label{tbl:kebutuhan-nonfungsional} \\
\hline
\textbf{Kode} & \textbf{Kebutuhan Non-Fungsional} & \textbf{Deskripsi} \\
\hline
\endfirsthead

\multicolumn{3}{c}%
{{\tablename\ \thetable{} -- lanjutan dari halaman sebelumnya}} \\
\hline
\textbf{Kode} & \textbf{Kebutuhan Non-Fungsional} & \textbf{Deskripsi} \\
\hline
\endhead

\hline
\multicolumn{3}{r}{\small\textit{Berlanjut ke halaman berikutnya...}} \\
\endfoot

\hline
\endlastfoot

% ISI TABEL
NF-01 & \textit{Understandability} &
Output narasi dan visualisasi harus menggunakan bahasa dan format yang mudah
dimengerti oleh pengguna. \\
\hline

NF-02 & \textit{Performance Efficiency} &
Sistem (terutama modul generasi narasi) harus mampu menghasilkan respons
dalam waktu yang wajar (< 1 menit) setelah data diinput. \\
\hline

NF-03 & \textit{Reproducibility} &
Eksperimen model dan kode program harus dapat dijalankan ulang dengan hasil
yang konsisten untuk memastikan validitas ilmiah. \\
\hline

NF-04 & \textit{Maintainability} &
Kode program harus terstruktur dengan baik, terdokumentasi dengan jelas, dan mudah dimodifikasi untuk perbaikan atau pengembangan. \\
\hline

NF-05 & \textit{Scalability} &
Sistem harus dapat dikembangkan menjadi lebih besar lagi di kemudian hari
dengan mudah. \\
\hline

\end{longtable}
