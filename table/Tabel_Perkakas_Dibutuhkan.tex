% ==========================================
% TABEL PERKAKAS YANG DIBUTUHKAN
% ==========================================
\begin{longtable}{|p{1cm}|p{4cm}|p{8cm}|}
\caption{Perkakas yang Dibutuhkan}
\label{tbl:perkakas-dibutuhkan} \\
\hline
\textbf{No} & \textbf{Perkakas} & \textbf{Deskripsi} \\
\hline
\endfirsthead

\multicolumn{3}{c}%
{{\tablename\ \thetable{} -- lanjutan dari halaman sebelumnya}} \\
\hline
\textbf{No} & \textbf{Perkakas} & \textbf{Deskripsi} \\
\hline
\endhead

\hline \multicolumn{3}{r}{\small\textit{Berlanjut ke halaman berikutnya...}} \\
\endfoot

\hline
\endlastfoot

1 & \textit{Python} & Bahasa pemrograman utama untuk pengembangan model, analisis data, dan integrasi sistem. \\
\hline
2 & \textit{Pandas} & \textit{Library} untuk manipulasi data tabular dan persiapan dataset. \\
\hline
3 & \textit{NumPy} & \textit{Library} untuk operasi numerik. \\
\hline
4 & \textit{Matplotlib} & \textit{Library} dasar untuk pembuatan visualisasi grafik. \\
\hline
5 & \textit{Seaborn} & \textit{Library} visualisasi data statistik tingkat tinggi yang dibangun di atas \textit{Matplotlib}. \\
\hline
6 & \textit{Statsmodels} & \textit{Library} untuk analisis statistik dan pemodelan ekonometrika (VAR) serta \textit{time-series} klasik (ARIMA). \\
\hline
7 & \textit{Scikit-learn} & \textit{Library} untuk \textit{preprocessing}, metrik evaluasi, dan implementasi model \textit{machine learning} (Random Forest). \\
\hline
8 & \textit{XGBoost} & \textit{Library} khusus untuk implementasi algoritma XGB. \\
\hline
9 & \textit{TensorFlow} & \textit{Framework} untuk pengembangan model \textit{deep learning} sekuensial (LSTM dan Bi-LSTM). \\
\hline
10 & \textit{SHAP} & \textit{Library} untuk interpretasi model secara global dan lokal berbasis \textit{Game Theory}. \\
\hline
11 & \textit{LIME} & \textit{Library} untuk interpretasi model secara lokal menggunakan pendekatan perturbasi. \\
\hline
12 & \textit{OpenAI API} & Layanan API LLM untuk mengubah hasil interpretasi numerik (XAI) menjadi narasi teks otomatis. \\
\hline
13 & \textit{Git} dan \textit{GitHub} & Sistem kontrol versi untuk manajemen kode sumber dan kolaborasi. \\
\hline

\end{longtable}
