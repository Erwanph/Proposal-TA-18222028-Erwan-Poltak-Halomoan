% ==========================================
% TABEL LINIMASA PENGERJAAN TUGAS AKHIR (LONGTABLE)
% ==========================================
\begin{longtable}{|p{4cm}|p{9cm}|}
\caption{Linimasa Pengerjaan Tugas Akhir}
\label{tbl:linimasa-pengerjaan} \\
\hline
\textbf{Bulan} & \textbf{Kegiatan} \\
\hline
\endfirsthead

\multicolumn{2}{c}%
{{\tablename\ \thetable{} -- lanjutan dari halaman sebelumnya}} \\
\hline
\textbf{Bulan} & \textbf{Kegiatan} \\
\hline
\endhead

\hline \multicolumn{2}{r}{\small\textit{Berlanjut ke halaman berikutnya...}} \\
\endfoot

\hline
\endlastfoot

September--Oktober 2025 & Identifikasi permasalahan, eksplorasi literatur pendukung, serta menentukan pendekatan dan metode Tugas Akhir yang akan digunakan. \\
\hline
November--Desember 2025 & Pengumpulan dan persiapan dataset, perangkat, dan teknologi yang diperlukan, serta penyusunan proposal tugas akhir. \\
\hline
Januari--Februari 2026 & Pengembangan sistem utama dan integrasi teknologi sesuai metode yang telah dirancang. \\
\hline
Maret--April 2026 & Pengujian sistem dan analisis hasil eksperimen. \\
\hline
Mei 2026 & Penyempurnaan sistem dan laporan berdasarkan temuan pengujian serta arahan dari pembimbing. \\
\hline
Juni 2026 & Penyelesaian laporan tugas akhir. \\
\hline

\end{longtable}
