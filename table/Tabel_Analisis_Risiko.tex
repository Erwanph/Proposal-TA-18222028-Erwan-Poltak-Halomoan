% ==========================================
% TABEL ANALISIS RISIKO DAN MITIGASI
% ==========================================
\begin{table}[H]
\centering
\caption{Analisis Risiko dan Mitigasi}
\label{tbl:analisis-risiko}
\begin{tabular}{|p{0.5cm}|p{4cm}|p{4cm}|p{5cm}|}
\hline
\textbf{No} & \textbf{Risiko} & \textbf{Dampak} & \textbf{Mitigasi} \\
\hline
1 & Data target prediksi terbatas  & Akurasi model rendah dan \textit{overfitting} pada model \textit{deep learning}. & 
\begin{enumerate}[leftmargin=*, nosep, topsep=0pt]
\item Menggunakan \textit{time-series augmentation} (\textit{sliding window}) untuk meningkatkan sampel pelatihan.
\item Implementasi \textit{regularization} (\textit{dropout}, L1/L2) pada model Bi-LSTM.
\item Melakukan \textit{cross-validation} untuk validasi hasil prediksi.
\end{enumerate} \\
\hline
2 & Akurasi model \textit{deep learning} rendah & Validasi model \textit{deep learning} mungkin tidak optimal dibanding \textit{machine learning} atau model ekonometrika tradisional. & 
\begin{enumerate}[leftmargin=*, nosep, topsep=0pt]
\item Melakukan \textit{tuning hyperparameter} untuk mencari kombinasi parameter optimal yang menghasilkan kinerja model terbaik.
\item Melakukan proses \textit{feature engineering} yang lebih luas lagi.
\end{enumerate} \\
\hline
\end{tabular}
\end{table}
