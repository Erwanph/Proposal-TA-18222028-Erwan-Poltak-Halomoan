% ==========================================
% TABEL DESKRIPSI DATASET HARIAN (MASIH HARUS DIREVISI)
% ==========================================
\begin{table}[H]
\centering
\caption{Deskripsi Kolom Dataset \textit{Timeframe} Bulanan}
\label{tbl:deskripsi-dataset-bulanan}
\begin{tabular}{|p{1cm}|p{3.5cm}|p{2cm}|p{6cm}|}
\hline
\textbf{No} & \textbf{Nama Kolom} & \textbf{Tipe} & \textbf{Deskripsi} \\
\hline
1 & ID & Object & Identitas unik setiap observasi bulanan \\
\hline
2 & Bulan & Integer & Bulan observasi (1--12) \\
\hline
3 & Tahun & Integer & Tahun observasi \\
\hline
4 & Inflation\_YoY\_Pct & Float & Tingkat inflasi tahunan (\%) \\
\hline
5 & Inflation\_Gap & Float & Selisih antara target inflasi dengan nilai aktualnya \\
\hline
6 & GDP\_Q\_forwardfilled & Float & Pertumbuhan PDB tahunan (\%) \\
\hline
7 & USD\_IDR\_monthly\_avg & Float & Kurs rata-rata bulanan Rupiah terhadap Dolar AS \\
\hline
8 & M2\_Triliun\_Rp & Float & Jumlah uang beredar (M2) dalam triliun Rupiah \\
\hline
9 & IHSG\_end\_of\_month & Float & Harga IHSG pada penutupan perdagangan bulan tersebut \\
\hline
10 & Credit\_Growth & Float & Total kredit perbankan dalam triliun Rupiah \\
\hline
11 & Consumer\_Confidence & Float & Tingkat keyakinan konsumen pada situasi ekonomi \\
\hline
12 & Federal\_Funds\_Rate\_Pct & Float & Suku bunga acuan Federal Reserve (\%) \\
\hline
13 & VIX\_Volatility\_Index & Float & Indeks volatilitas pasar keuangan global (VIX) \\
\hline
14 & Oil\_Price\_Brent & Float & Harga Minyak Bumi Dunia dalam Rupiah \\
\hline
15 & Reserves & Float & Cadangan devisa negara \\
\hline
16 & BI\_Rate(\%) & Float & Suku bunga acuan Bank Indonesia (\%) \\
\hline
\end{tabular}
\end{table}
