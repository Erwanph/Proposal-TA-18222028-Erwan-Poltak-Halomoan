% ==========================================
% BAB IV DESAIN KONSEP SOLUSI
% ==========================================
\chapter{DESAIN KONSEP SOLUSI}
\label{chap:desain-konsep-solusi}

Pada bab ini dijelaskan desain konseptual solusi yang diusulkan 
sebagai solusi atas permasalahan yang ada dalam sistem saat ini. 
Setiap tahapan dalam desain ini dirancang sejalan dengan metodologi CRISP-DM 
yang digunakan dalam tugas akhir ini.

% --- Desain Solusi ---
\section{Desain Solusi}
\label{sec:desain-solusi}

Tugas akhir ini akan dikerjakan sesuai dengan ajuan desain konseptual 
yang dapat dilihat pada Gambar \ref{fig:desain-solusi}. Diagram 
ini menggambarkan alur proses pengerjaan tugas akhir dengan lengkap, 
mulai dari tahap \textit{data collection}, \textit{data understanding}, \textit{data preparation}, \textit{modelling}, 
hingga tahap penjelasan hasil prediksi menggunakan integrasi XAI dan LLM, 
serta evaluasi kuantiatif dan interpretabilitas.

\begin{figure}[H]
  \centering
  \captionsetup{justification=centering}
  \includegraphics[width=1.08\textwidth]{image/perbandingan_sistem.png}
  \caption{Desain Solusi Pengerjaan Tugas Akhir}
  \label{fig:desain-solusi}
\end{figure}

Perbedaan mendasar antara sistem yang ada saat ini 
dengan desain solusi yang diajukan dapat dilihat secara rinci pada 
Tabel \ref{tbl:perbandingan-alternatif}.

\input table/Tabel_Perbandingan_Sistem.tex

Sistem yang diajukan dalam tugas akhir ini menjadi solusi atas 
kelemahan yang ada pada sistem prediksi saat ini. Sistem yang diajukan
memiliki aspek transparansi yang lebih baik melalui integrasi metode
XAI dan LLM untuk menjelaskan hasil prediksi, serta cakupan data
yang lebih luas pada indikator makroekonomi lokal dan ditambahkan dengan
indikator global.

Perbedaan utama terletak pada proses pasca pemodelan (\textit{post-modelling}). 
Sistem ajuan tidak berhenti hanya pada \textit{output} prediksi, tetapi juga 
mengintegrasikan XAI yang kemudian dinarasikan oleh LLM untuk memudahkan pemahaman terhadap faktor yang 
memengaruhi hasil prediksi bagi orang awam. Mekanisme ini memungkinkan 
sistem untuk tidak hanya memberikan hasil prediksi suku bunga ke depan, 
tetapi juga menjelaskan alasan dibalik hasil prediksi tersebut berdasarkan 
dinamika variabel ekonomi yang memengaruhinya. Selain itu, proses pra-pemrosesan 
data pada solusi yang diusulkan dirancang jauh lebih terstruktur 
dengan tahapan \textit{cleaning}, \textit{imputation}, dan \textit{feature engineering} 
untuk menangani \textit{raw dataset} yang kompleks. Hal ini bertujuan 
untuk memastikan kualitas data yang lebih baik sebelum masuk ke tahap 
pemodelan.