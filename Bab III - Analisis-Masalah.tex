% ==========================================
% BAB III ANALISIS MASALAH
% ==========================================
\chapter{ANALISIS MASALAH}
\label{chap:analisis-masalah}

Pada bab ini dibahas terkait permasalahan penetapan suku bunga acuan 
Bank Indonesia, analisis kondisi sistem saat ini, identifikasi 
kebutuhan pengguna, serta analisis pemilihan solusi yang tepat untuk 
mengatasi permasalahan tersebut. Analisis kondisi ini dilakukan dengan 
mengacu pada tahap awal CRISP-DM yang telah dibahas pada Bab I, 
yaitu \textit{business understanding} dan \textit{data understanding}.

% --- Analisis Kondisi Saat Ini ---
\section{Analisis Kondisi Saat Ini}
\label{sec:kondisi-saat-ini}

Pada subbab ini dibahas terkait model konseptual sistem penetapan suku 
bunga acuan Bank Indonesia saat ini, serta masalah yang ada pada sistem 
tersebut. Analisis kondisi ini dilakukan dengan mengacu pada tahap awal 
CRISP-DM yang telah dibahas pada Bab I, yaitu \textit{business understanding}. 
Tahap ini memberikan pemahaman terkait konteks masalah yang akan 
diselesaikan dalam tugas akhir ini.

\subsection{Business Understanding}
Penetapan BI-Rate merupakan salah satu bentuk kebijakan moneter utama 
Bank Indonesia untuk mencapai target inflasi dan menjaga stabilitas 
nilai tukar. Berdasarkan analisis kondisi saat ini, kebutuhan bisnis 
yang mendesak adalah transparansi dan edukasi publik.

Meskipun data ekonomi tersedia secara luas, proses pengambilan 
keputusan di balik penetapan BI-Rate sering kali dipersepsikan 
sebagai \textit{black box}. Publik mengetahui data masukan 
(inflasi, PDB, kurs) dan hasil keluaran (BI-Rate), tetapi tidak 
memahami mekanisme internal atau bobot kontribusi setiap variabel 
terhadap keputusan tersebut.

Oleh karena itu, diperlukan sebuah sistem yang mampu menjembatani 
kesenjangan antara data kuantitatif yang kompleks dengan pemahaman 
masyarakat. Sistem ini harus mampu memprediksi arah kebijakan BI-Rate 
dengan akurasi yang optimal sekaligus membuka \textit{black box} 
tersebut melalui penjelasan yang transparan. Tujuan akhirnya adalah 
meningkatkan akuntabilitas kebijakan, mengurangi asimetri informasi, 
dan membantu membentuk ekspektasi pasar yang lebih rasional melalui 
narasi berbasis data yang mudah dipahami.

\subsection{Gambaran Sistem Saat Ini}
Proses penetapan BI-Rate yang berjalan saat ini melibatkan analisis 
mendalam dan prosedur bertingkat di internal Bank Indonesia.

\begin{figure}[h]
  \centering
  \captionsetup{justification=centering}
  \includegraphics[width=0.9\textwidth]{image/proses_penetapan_birate.png}
  \caption{Proses Penetapan BI-Rate Saat Ini}
  \label{fig:proses-penetapan-birate}
\end{figure}

Sebagaimana diilustrasikan pada Gambar \ref{fig:proses-penetapan-birate}, 
proses dimulai dari pengumpulan data ekonomi makro domestik dan global. 
Departemen terkait, khususnya Departemen Kebijakan Ekonomi dan Moneter (DKEM), 
melakukan analisis dan simulasi model ekonometrika untuk menyusun 
rekomendasi kebijakan. Hasil analisis ini dibahas dalam Rapat Dewan 
Gubernur (RDG) yang berlangsung selama dua hari: hari pertama untuk 
evaluasi kondisi ekonomi dan hari kedua untuk penetapan 
kebijakan \autocite{bi2023moneter}.

Setelah keputusan ditetapkan, Bank Indonesia mengumumkan hasil RDG 
melalui siaran pers resmi. Namun, terdapat keterbatasan dalam sistem 
komunikasi saat ini:
\begin{enumerate}
    \item \textbf{Penjelasan Kualitatif} 
    
    Narasi dalam siaran pers umumnya bersifat makro dan kualitatif. Tidak ada rincian kuantitatif eksplisit yang menyatakan seberapa besar bobot kontribusi spesifik dari setiap indikator (misalnya berapa persen kenaikan inflasi memengaruhi keputusan kenaikan suku bunga bulan ini).
    
    \item \textbf{Kesenjangan Informasi} 
    
    Pengguna awam dapat kesulitan memvalidasi rasionalitas keputusan penetapan BI-Rate secara mandiri karena ketiadaan alat bantu analisis yang transparan.
\end{enumerate}

Selain sistem internal BI, terdapat juga berbagai model prediksi yang 
dikembangkan oleh penelitian terdahulu, seperti yang ditunjukkan pada 
Gambar \ref{fig:sistem-prediksi-terdahulu}.

\begin{figure}[H]
  \centering
  \captionsetup{justification=centering}
  \includegraphics[width=1\textwidth]{image/sistem_prediksi_literatur.png}
  \caption{Gambaran Sistem Prediksi BI-Rate Terdahulu \autocite{putri2019prediksi}}
  \label{fig:sistem-prediksi-terdahulu}
\end{figure}

Sistem prediksi terdahulu hanya fokus pada akurasi prediksi. Alurnya 
bersifat linear, mulai dari \textit{input} data, normalisasi, pemodelan, 
hingga \textit{output} hasil prediksi. Sistem ini belum memiliki fitur 
interpretabilitas (\textit{explainability}). Pengguna hanya menerima 
angka hasil prediksi tanpa mengetahui kenapa angka tersebut muncul, 
sehingga sistem ini kurang efektif untuk tujuan edukasi dan analisis 
kebijakan.

% --- Analisis Kebutuhan ---
\section{Analisis Kebutuhan}
\label{sec:analisis-kebutuhan}

Pada subbab ini dibahas kebutuhan spesifik yang harus dipenuhi oleh 
sistem untuk menjawab permasalahan pengguna yang telah diidentifikasi. 
Analisis ini mencakup identifikasi masalah dari sudut pandang pengguna, 
serta spesifikasi kebutuhan fungsional dan non-fungsional sistem yang 
akan dikembangkan.

\subsection{Identifikasi Masalah Pengguna}
Berdasarkan observasi, masalah utama yang dihadapi pengguna adalah:
\begin{enumerate}
\item Pengguna awam sulit memahami hubungan sebab-akibat yang kompleks antara fluktuasi indikator ekonomi dengan perubahan BI-Rate tanpa bantuan alat analisis kuantitatif.
\item Pengguna tidak memiliki akses ke alat bantu yang dapat memberikan analisis naratif mengenai kondisi ekonomi terkini, sehingga sering kali bergantung pada opini pengamat yang subjektif atau tertunda (\textit{lagging}).
\end{enumerate}

\subsection{Kebutuhan Fungsional}
Kebutuhan fungsional mendefinisikan fitur dan kemampuan spesifik yang 
harus dimiliki oleh sistem yang akan dikembangkan. Rincian kebutuhan 
fungsional dapat dilihat di Tabel \ref{tbl:kebutuhan-fungsional}.

\input table/Tabel_Kebutuhan_Fungsional.tex

\subsection{Kebutuhan Non-Fungsional}
Kebutuhan non-fungsional mendefinisikan batasan kualitas sistem. 
Rincian kebutuhan non-fungsional dapat dilihat di Tabel \ref{tbl:kebutuhan-nonfungsional}.

\input table/Tabel_Kebutuhan_NonFungsional.tex

% --- Analisis Pemilihan Solusi ---
\section{Analisis Pemilihan Solusi}
\label{sec:analisis-pemilihan-solusi}

Pada subbab ini dilakukan analisis terhadap alternatif pendekatan 
solusi untuk memenuhi kebutuhan sistem, serta pemilihan solusi terbaik.
Analisis ini menggunakan pendekatan \textit{comparative selection}, 
yaitu berbagai kandidat metode dievaluasi berdasarkan karakteristik 
data, serta kebutuhan pengguna akan transparansi.

Berikut adalah analisis pemilihan metode untuk tiga komponen utama 
sistem, yaitu model prediksi, XAI, dan LLM.

\subsection{Seleksi Model Prediksi}
Tantangan utama dalam data deret waktu ekonomi adalah menangkap 
pola linier. Bagian ini akan menganalisis kandidat model dari 
ekonometrika tradisional, \textit{machine learning}, dan \textit{deep learning}. 

\input table/Tabel_Seleksi_Model.tex

Berdasasarkan analisis pada Tabel \ref{tbl:seleksi-model}, seluruh 
kategori model memiliki keunggulan masing-masing. Oleh karena itu,
strategi yang diambil adalah melakukan komparasi performa antara 
model ekonometrika (ARIMA, VAR), ML (Random Forest, XGBoost), dan 
DL (LSTM, Bi-LSTM). Pendekatan ini memastikan bahwa eksperimen nanti 
dapat mengidentifikasi model terbaik yang paling sesuai dengan 
karakteristik data makroekonomi yang kompleks.

\subsection{Seleksi Metode XAI}
Komponen ini bertujuan untuk menginterpretasikan model \textit{black box} agar 
transparan dan dapat diketahui faktor apa saja yang memengaruhi hasil 
prediksinya. Tantangan terbesar dalam data ekonomi adalah 
multikolinieritas yang ada antar variabel. Metode XAI yang dipilih 
harus dapat menangani kondisi ini.

\input table/Tabel_Seleksi_XAI.tex

Berdasarkan analisis pada Tabel \ref{tbl:seleksi-xai}, metode XAI yang 
digunakan pada tugas akhir ini adalah SHAP dan LIME untuk meminimalisir 
bias interpretasi pada data yang mengalami multikolinearitas. SHAP 
memberikan penjelasan yang konsisten dan matematis dengan latar belakang 
\textit{game theory} dengan tiga sifatnya (\textit{local accuracy}, 
\textit{missingness}, dan \textit{consistency}). LIME 
memberikan penjelasan lokal yang mudah dipahami melalui aproksimasi linear 
lokal. Keduanya lebih \textit{robust} dibanding PFI dalam menangani fitur-fitur 
yang berkorelasi tinggi karena SHAP memperhitungkan semua kombinasi 
subset fitur secara eksplisit dan LIME bekerja pada representasi lokal 
yang dapat disesuaikan dengan struktur data.

\subsection{Seleksi Model LLM}
Komponen terakhir ini (LLML) akan membuat narasi kebijakan yang mudah dipahami 
awam berdasarkan hasil prediksi dan XAI. Kriteria utamanya adalah 
kemampuan penalaran (\textit{reasoning}) model dan minimnya halusinasi.

\input table/Tabel_Seleksi_LLM.tex

Berdasarkan analisis pada Tabel \ref{tbl:seleksi-llm}, model LLM yang 
digunakan pada tugas akhir ini adalah OpenAI untuk memprioritaskan 
akurasi penjelasan di atas efisiensi biaya, mengingat kesalahan 
interpretasi dalam konteks keuangan dapat berdampak fatal terhadap 
kredibilitas sistem.

\subsection{Penentuan Solusi}
\label{subsec:penentuan-solusi}

Berdasarkan analisis seleksi komponen yang telah dilakukan pada 
sub-bab sebelumnya, desain solusi akhir sistem prediksi BI-Rate 
dalam Tugas Akhir ini ditetapkan menggunakan pendekatan 
arsitektur \textit{explainable artificial intelligence with generative narrative}.

Keputusan ini diambil dengan pertimbangan utama untuk menyeimbangkan 
akurasi prediksi dengan kebutuhan interpretasi yang menjadi masalah 
yang ingin diselesaikan. Secara ringkas, arsitektur solusi terpilih 
terdiri dari:

\begin{enumerate}
    \item \textbf{Modul Prediksi} 
    
    Tidak mengandalkan satu model tunggal, 
    melainkan menggunakan strategi komparasi performa antara model 
    tradisional, ML, dan DL. Pendekatan ini dapat memastikan bahwa 
    eksperimen nanti dapat mengidentifikasi model terbaik yang 
    paling sesuai dengan karakteristik data makroekonomi yang kompleks.
    
    \item \textbf{Modul Interpretasi} 
    
    Mengintegrasikan metode 
    SHAP sebagai penjelas utama untuk menjamin konsistensi atribusi 
    fitur secara global, lalu didukung oleh LIME untuk validasi penjelasan 
    lokal pada kasus prediksi yang spesifik. Keduanya dipilih 
    untuk memitigasi risiko misinterpretasi akibat multikolinearitas 
    data makroekonomi yang ada.
    
    \item \textbf{Modul Narasi} 
    
    Menggunakan OpenAI sebagai LLM untuk 
    mengubah \textit{output} hasil prediksi dan XAI menjadi teks 
    narasi kebijakan yang mudah dipahami dan menjembatani kesenjangan 
    teknis antara sistem AI dan pengguna awam.
\end{enumerate}

Integrasi ketiga komponen ini diharapkan mampu menghasilkan sistem 
yang optimal dalam memprediksi BI-Rate, serta transparan dan 
komunikatif, sesuai dengan tujuan utama tugas akhir ini.

% --- Data Understanding ---
\section{Data Understanding}
\label{sec:data-understanding}

Pada tahap ini, dilakukan eksplorasi dan analisis awal terhadap data 
makroekonomi yang akan digunakan untuk membuat model prediksi BI-Rate. 
Dataset yang digunakan dalam tugas akhir ini terdiri dari 11 variabel 
dengan frekuensi bulanan, mulai dari Juli 2005 hingga September 
2025. Total data yang diobservasi adalah sebanyak 243 data point.

\subsection{Data Overview}
Dataset terdiri atas 15 kolom pada Tabel \ref{tbl:deskripsi-dataset-bulanan}.

\input table/Tabel_Dataset_Bulanan.tex

